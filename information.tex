%%%%%%%%%%%%%%%%%%%%%%%%%%%%%%%%%%%%%%%%%%%%%%%%%%%%%%%%%%%%%%%%%%%%%%%
%%%%%%%%% Aşağıda istenilen bilgileri dikkatlice doldurunuz.   %%%%%%%%
%%%%%%%%% Doldurmanız istenilen ifadenin sonunda TR ya da EN   %%%%%%%%
%%%%%%%%% yazıyorsa, sırasıyla Türkçe veya İngilizce olarak    %%%%%%%%
%%%%%%%%% doldurunuz. Eğer herhangi bir ifade yoksa, tezinizi  %%%%%%%%
%%%%%%%%% hangi dilde yazıyorsanız (Türkçe veya İngilizce), o  %%%%%%%%
%%%%%%%%% dile göre doldurunuz. İsimleri yazarken soyisimleri  %%%%%%%%
%%%%%%%%% büyük harf ile yazınız.                              %%%%%%%%
%%%%%%%%%%%%%%%%%%%%%%%%%%%%%%%%%%%%%%%%%%%%%%%%%%%%%%%%%%%%%%%%%%%%%%%
%%%%%%%%%%%%%%%%%%%%%%%%%%%%%%%%%%%%%%%%%%%%%%%%%%%%%%%%%%%%%%%%%%%%%%%

% Tezin türü: Bitirme Tezi/Graduation Thesis ve Proje/Project
\def\typeofThesisTR{Bitirme Tezi}
\def\typeofThesisEN{Graduation Thesis}

% Tez başlığını Türkçe olarak yazınız. Tez başlıklarında tire (-) kullanmayınız.
\def\titleTR{Protein Yapısına Dayalı Sanal Tarama Yaklaşımına Alternatif Kimyasal Yapı Vektörü Bazlı Derin Öğrenme Modeli}

% Tez başlığını İngilizce olarak yazınız. Tez başlıklarında tire (-) kullanmayınız.
\def\titleEN{A Deep Learning Model Using Chemical Structure Embedding as an Alternative to Structure-Based Virtual Screening}

% İsminizi yazınız.
\def\student{Birinci İsim SOYİSİM}
\def\studentno{1505A000}

\def\studentii{İkinci İsim SOYİSİM}
\def\studentiino{1605A000}

\def\studentiii{}
\def\studentiiino{}

\def\studentiv{}
\def\studentivno{}

% fakülte adını İngilizce ve türkçe yazınız.
\def\facultyEN{Faculty of Chemical and Metallurgical Engineering}
% Anabilim dalınızın Türkçe adını yazınız.
\def\facultyTR{Kimya-Metalurji Fakültesi}

% Anabilim dalınızın İngilizce adını yazınız.
\def\departmentEN{Department of Biongineering}
% Anabilim dalınızın Türkçe adını yazınız.
\def\departmentTR{Biyomühendislik Bölümü}
% Programınızın İngilizce adını yazınız.
\def\programEN{Biongineering}
% Programınızın Türkçe adını yazınız.
\def\programTR{Biyomühendislik Ana Bilim Dalı}
% Tez/proje imza tarihini yazınız. (gg.aa.yyyy)
\def\dateFull{15/06/2021}
% Tez sınavı tarihini, tez için kullandığınız dilde şu formatta yazınız. (ay, yıl). Ayrıca yıl bilgisini de yazınız.
\def\date{June, 2021}
\def\Year{2021}

% Tezinizde eş-danışman varsa 1, yoksa 0 yazınız.
\def\isThereCoAdvisor{0}

% Tez danışmanınızın ismini Türkçe ünvanı ile yazınız.
\def\advisorTR{Dr.Öğr.Üyesi Alper Yılmaz}
% Tez danışmanınızın ismini İngilizce ünvanı ile yazınız.
\def\advisorEN{Asst.Prof.Dr. Alper YILMAZ}
% Tez danışmanınızın bağlı olduğu kurumu tez için kullandığınız dilde yazınız.
\def\advisorUni{Yildiz Technical University}

% Eş-danışmanınız varsa ismini Türkçe ünvanı ile yazınız. Yoksa bu kısmı atlayınız.
\def\coadvisorTR{Doç. Dr. }
% Eş-danışmanınız varsa ismini İngilizce ünvanı ile yazınız. Yoksa bu kısmı atlayınız.
\def\coadvisorEN{Assoc. Prof. Dr. }
% Eş-danışmanınızın bağlı olduğu kurumu tez için kullandığınız dilde yazınız.
\def\coadvisorUni{Yildiz Technical University}

% imza metni
%% EN
\def\imzaEN{This \textbf{\MakeUppercase\typeofThesisEN} submitted by \student~is approved by the committee on \dateFull~in in Yildiz Technical University, Faculty of Chemical and Metallurgical Engineering, Department of Bioengineering.}
%% if two students
%% ..submitted by \student~and \studentii~is approved..
%% if three students
%% ..submitted by \student, \studentii~and \studentiii~is approved..

%% TR
\def\imzaTR{\student~tarafından hazırlanan bitirme tezi çalışması \dateFull~tarihinde aşağıdaki jüri tarafından Yıldız Teknik Üniversitesi Kimya-Metalürji Fakültesi Biyomühendislik Bölümü’nde \textbf{\MakeUppercase\typeofThesisTR} olarak kabul edilmiştir.}
%% iki öğrenci varsa
%% \student~ve \studentii~tarafından hazırlanan..
%% üç öğrenci varsa
%% \student, \studentii ve \studentiii~tarafından hazırlanan..

% Aşağıdaki kısma sırası ile tez sınavı üyelerinin isimlerini ünvanları ile 
% birlikte yazınız. Ardınan kişilerin bağlı bulundukları kurumları tez için 
% kullandığınız dilde yazınız. Yüksek lisans için ilk 2, doktora için ilk 4 
% bilgiyi doldurmanız gereklidir.

\def\memberi{Prof. Dr. Dilek TURGUT-BALIK}
\def\memberiUni{Yildiz Technical University}

\def\memberii{Assoc. Prof. Dr. Emrah Şefik ABAMOR}
\def\memberiiUni{Yildiz Technical University}


%% \def\memberiii{}
%% \def\memberiiiUni{}

%% \def\memberiv{}
%% \def\memberivUni{}

\def\acknowledgementText{
    % Buraya teşekkür metninizi tez için kullandığınız dilde yazınız. 
Foremost, we are grateful to have completed our thesis under the supervision of Asst. Prof. Alper Yılmaz who has supported us and educated us on the topic of our project and has inspired us to work at our best.


Besides our professor, we would like to thank our friends and families for their moral support and understanding. They have done their best to help us on matters that they can and we are grateful. 

}

\def\abstractTextEnglish{
    % Buraya İngilizce olarak tez özetini yazınız.
Virtual screening is a branch of in-silico drug design and discovery and it is involved with searching through chemical libraries computationally, to obtain lead compounds. Recently, various methods have been emerging to advance this field and more specifically deep learning applications have started to gain popularity. Deep learning is a subfield of machine learning that has the ability to learn from enormous data very rapidly, making it a desirable application for many fields of science. 

Structure-based virtual screening methods mainly perform molecular docking of molecules one by one to a target molecule making this process rather time-consuming and impractical and thus can become costly. The purpose of this thesis is to employ deep learning to overcome such issues. 

Initially, we have downloaded a data set from PDBbind that contains proteins and ligands in docked conformation. Then, we have generated the cavities on the proteins by using the IChem toolkit while embedding the ligands using an unsupervised model that is called Mol2vec. Once both the data on cavities and ligands are generated, the deep neural network was constructed to ultimately find a ligand that would bind to the given cavity.    

In conclusion, our model has successfully predicted molecular structures that were very similar to already known molecules that reside in the specified cavity with 95\% accuracy.  

}




\def\abstractKeywordsEnglish{
    % Buraya İngilizce olarak tez için geçerli anahtar kelimeleri yazınız
    Virtual screening, deep learning, machine learning, protein and ligand interactions, artificial neural networks, docking, bioinformatics.
}

\def\abstractTextTurkish{
  Sanal tarama, in-silico ilaç tasarımı ve keşfinin bir dalıdır ve lider bileşikleri elde etmek için kimyasal kütüphanelerde hesaplama yoluyla araştırmaya dahil edilir. Bu alanı ilerletmek için birçok yöntem vardır ve bu yöntemlerden biri olan derin öğrenme uygulamaları günümüzde popülerlik kazanmaya başlamıştır. Derin öğrenme, oldukça büyük verilerden çok hızlı bir şekilde öğrenme yeteneği ile birçok bilim alanında tercih edilen, makine öğreniminin bir alt alanıdır. 

  Yapı bazlı sanal tarama metodları genellikle moleküllerin teker teker kenetlenmesini gerçekleştirerek en iyi bağlanan bileşikleri ortaya çıkarır, ancak bu işlem zaman harcayan, pratik olmayan, maliyetli bir işlemdir. Bu tezin amacı, derin öğrenme yaklaşımını kullanarak bu gibi sorunların üstesinden gelmektir.

  Öncelikle, PDBbind veritabınından, kenetlenmiş konformasyonda proteinler ve ligantlar içeren bir veri seti indirildi. Daha sonra, ligantlar denetimsiz bir model olan Mol2vec aracılığıyla vektör haline getirildi ve IChem yazılımı kullanarak proteinler üzerindeki kaviteler tespit edildi. Hem kaviteler hem de ligantlar üzerindeki veriler elde edildikten sonra, derin sinir ağı sonuçta verilen kaviteye bağlanacak bir ligant bulmak için yapılandırıldı.
  
  Sonuç olarak, oluşturulmuş model başarılı bir şekilde moleküler yapıyı tahmin ettiği ve tahmin edilen ligant molekül yapıları, halihazırda bağlanan ligant moleküleriyle yakın benzerlik taşıdığı görüldü. Oluşturulan modelin doğruluğu \%95 olarak tespit edildi.  
}

\def\abstractKeywordsTurkish{ Sanal tarama, derin öğrenme, makine öğrenmesi, protein ve ligant etkileşimleri, yapay sinir ağları, biyoenformatik 

}

% Tez için kullandığınız dilde, aşağıdaki kısma tez için aldığınız destekleri yazınız. Eğer destek almadıysanız küme parantezleri içerisindeki yazıları siliniz.

% \def\supports{This study was supported by the Scientific and Technological Research Council of Turkey (TUBITAK) Grant No: 2210.}


%%%%%%%%%%%%%%%%%%%%%%%%%%%%%%%%%%%%%%%%%%%%%%%%%%%%%%%%%%%%%%
%%%% Aşağıdaki alana "\item[sembol] Sembol_açıklaması" %%%%%%%
%%%% şeklinde sembollerinizi giriniz. Açıklamanın ilk  %%%%%%%
%%%% harfine göre sıralayınız. Eğer sembol kullanmı-   %%%%%%%
%%%% yorsanız "\def\symbols{}" olacak şekilde küme     %%%%%%%
%%%% parantezlerinin içini siliniz.                    %%%%%%%
%%%%%%%%%%%%%%%%%%%%%%%%%%%%%%%%%%%%%%%%%%%%%%%%%%%%%%%%%%%%%%

\def\symbols{

    \begin{abbrv}
        \item[Ai]               Activities of Daily Life
        \item[c]                Alternate Step Test
        \item[C]                Body Mass Index
        \item[CR]               Cross Step moving on Four Stops
        \item[$fc(.)$]          Dynamic Bayesian Networks
        \item[$\Delta H$]       Demura's Fall Risk Assessment Chart
        \item[$\lambda i$]      Electromyography
        \item[$\Omega$]         Faculdade de Engenharia da Universidade do Porto
    \end{abbrv}

}


%%%%%%%%%%%%%%%%%%%%%%%%%%%%%%%%%%%%%%%%%%%%%%%%%%%%%%%%%%%%%%
%%%% Aşağıdaki alana "\item[kısaltma] kısaltma_açıklaması" %%%
%%%% şeklinde kısaltmalarınızı giriniz. Kısaltmanın ilk    %%%
%%%% harfine göre sıralayınız. Eğer kısaltma kullanmıyor-  %%%
%%%% sanız "\def\abbrevations{}" olacak şekilde küme pa-   %%%
%%%% rantezlerinin içini siliniz.                          %%%
%%%%%%%%%%%%%%%%%%%%%%%%%%%%%%%%%%%%%%%%%%%%%%%%%%%%%%%%%%%%%%
\def\abbrevations{

    \begin{abbrv}
        \item[CBOW]             Continuous Bag Of Words
        \item[CNN]              Convolutional Neural Networks      
        \item[DTI]              Drug-Target Interactions
        \item[LBVS]             Ligand-Based Virtual Screening
        \item[LSTM]             Long Short Term Memory 
        \item[PDB]              Protein Data Bank
        \item[ReLU]             Rectified Linear Unit
        \item[RNN]              Recurrent Neural Networks
        \item[RCSB]             Research Collaboratory for Structural Bioinformatics
        \item[SBVS]             Structure-Based Virtual Screening
        \item[SMILES]           Simplified Molecular-Input Line-Entry System
        \item[SVM]              Support Vector Machine
        \item[3D]               Three Dimensional
        \item[3DGCN]            3D Graph Convolutional Network
    \end{abbrv}
    
}

\def\plagirismTextDeptEN{According to the maximum similarity rates stated in Yıldız Technical University Graduate School of Science and Engineering Thesis Originality Report Principles, this project study does not include any plagiarism; I declare that I accept all kinds of legal responsibilities that may arise in the probable situation to be determined and my student information given in this project is correct.}
\def\plagirismTextDeptTR{Yıldız Teknik Üniversitesi Fen Bilimleri Enstitüsü Tez Çalışması Orijinallik Raporu Uygulama Esasları’nda belirtilen azami benzerlik oranlarına göre bu proje çalışmasının herhangi bir intihal içermediğini; aksinin tespit edileceği muhtemel durumda doğabilecek her türlü hukuki sorumluluğu kabul ettiğimi ve bu projede verilen öğrenci bilgilerimin doğru olduğunu beyan ederim.}
